\documentclass[11pt]{article}
\usepackage[T1]{fontenc}
\usepackage[utf8]{inputenc}
\usepackage[margin=1in]{geometry}
\usepackage{mathpazo}
\linespread{1.05}
\usepackage{microtype}

\usepackage{xcolor}
\definecolor{KZblue}{HTML}{003C75}
\definecolor{KZlightblue}{HTML}{E6EEF7}
\definecolor{KZgray}{HTML}{4A4A4A}

\usepackage[colorlinks=true,linkcolor=KZblue,citecolor=KZblue,urlcolor=KZblue]{hyperref}

\usepackage{amsmath,amssymb,mathtools}

\usepackage{tcolorbox}
\tcbuselibrary{theorems,skins}
\newtcbtheorem
  [auto counter, number within=section]
  {theorem}{Theorem}%
  {colback=KZlightblue, colframe=KZblue, fonttitle=\bfseries\color{KZblue}, left=4pt, right=4pt, top=4pt, bottom=4pt}{th}

\begin{document}

\section{Randomized SVD (RSVD)}

\subsection{Conceptual Overview}
The core challenge in modern data analysis is dealing with matrices $A \in \mathbb{R}^{m \times n}$ that are too large to fit in memory or too computationally expensive to decompose using classical algorithms. The classical Singular Value Decomposition (SVD) requires $O(mn \min(m, n))$ floating-point operations. For massive datasets, this cubic scaling is prohibitive.

Randomized SVD (RSVD) rests on a profound geometric concentration phenomenon: Random vectors in high-dimensional spaces tend to be nearly orthogonal to any fixed low-dimensional subspace, but they capture the energy of the dominant subspace with high probability.

\subsubsection*{The "Range Finder" Intuition}
If a matrix $A$ has exact rank $k$, its columns span a $k$-dimensional subspace of $\mathbb{R}^m$. If we multiply $A$ by a set of $k$ random vectors $\omega_1, \dots, \omega_k$, the resulting vectors $y_i = A\omega_i$ will essentially be random linear combinations of the basis vectors of the column space of $A$. With probability 1 (for Gaussian vectors), these $y_i$ vectors will be linearly independent and will span the exact same column space as $A$.

However, real-world matrices are not exactly rank-$k$, but have a fast spectral decay. This means the singular values $\sigma_1, \sigma_2, \dots$ drop off quickly. In this scenario, applying $A$ to random vectors acts as a filter:
\[
y = A\omega = \sum_{i=1}^n \sigma_i u_i (v_i^T \omega)
\]
Since $\omega$ is random, the scalar projection $v_i^T \omega$ spreads energy across all directions roughly equally. However, the weighting by $\sigma_i$ amplifies the directions corresponding to the large singular values (the "signal") and suppresses the small singular values (the "noise"). Consequently, the vectors $Y = [y_1, \dots, y_\ell]$ align themselves with the dominant left singular vectors $u_1, \dots, u_k$.

\subsection{Algorithm Description}
The RSVD algorithm is a "matrix decomposition" technique that splits the problem into a probabilistic step (finding the subspace) and a deterministic step (exact SVD on a small matrix).

\subsubsection*{Inputs}
\begin{itemize}
    \item Matrix $A \in \mathbb{R}^{m \times n}$
    \item Target rank $k$
    \item Oversampling parameter $p$ (usually $5 \le p \le 10$).
    \item Total random vectors $\ell = k + p$.
\end{itemize}

\subsubsection*{Step 1: Gaussian Test Matrix Generation}
We draw a Gaussian random matrix $\Omega \in \mathbb{R}^{n \times \ell}$ such that every entry $\Omega_{ij} \sim \mathcal{N}(0, 1)$.

\textbf{Justification:} We choose the standard normal distribution because it is rotationally invariant. This means the algorithm's performance does not depend on the specific basis in which the input matrix $A$ is represented. Other distributions (like Rademacher variables $\pm 1$) work but require slightly more complex theoretical analysis.

\subsubsection*{Step 2: Sampling the Range (The Sketch)}
Compute the sketch matrix $Y$:
\[
Y = A\Omega \in \mathbb{R}^{m \times \ell}
\]
This step dominates the computational cost if implemented naively. In practice, this is a dense matrix-matrix multiplication (GEMM).

\textbf{Effect:} $Y$ is a "compressed" representation of the column space of $A$. If we define the SVD of $A = U\Sigma V^T$, then $Y \approx U_\ell \Sigma_\ell (V_\ell^T \Omega)$.

\subsubsection*{Step 3: Orthonormalization (Basis Generation)}
We compute the QR factorization of $Y$:
\[
Y = Q R
\]
where $Q \in \mathbb{R}^{m \times \ell}$ is a matrix with orthonormal columns, and $R$ is upper triangular.

\textbf{Crucial Detail:} We discard $R$. We only need $Q$.

\textbf{Why is this necessary?} Without orthonormalization, the columns of $Y$ would become increasingly collinear (parallel to the principal singular vector $u_1$) as the singular value gaps increase, leading to severe floating-point errors. $Q$ provides a numerically stable basis for the range of $Y$.

\subsubsection*{Step 4: Projection to Low-Dimension}
We form the projected matrix $B$:
\[
B = Q^T A \in \mathbb{R}^{\ell \times n}
\]
\textbf{Dimension Reduction:} We have reduced the dimension from $m$ (which could be millions) to $\ell$ (usually hundreds).

\textbf{Approximation:} Since $Q$ captures the range of $A$, $A \approx Q Q^T A = Q B$.

\subsubsection*{Step 5: Deterministic SVD}
Compute the SVD of the small matrix $B$:
\[
B = \tilde{U} \Sigma V^T
\]
where $\tilde{U} \in \mathbb{R}^{\ell \times \ell}$ and $V \in \mathbb{R}^{n \times n}$.

\textbf{Efficiency:} This step is extremely fast because $\ell$ is small. $O(\ell^2 n)$.

\subsubsection*{Step 6: Lifting (Reconstruction)}
We construct the approximate left singular vectors of $A$ by transforming $\tilde{U}$ back to the original coordinate system using $Q$:
\[
U = Q \tilde{U} \in \mathbb{R}^{m \times \ell}
\]
The approximate SVD is then:
\[
A \approx U \Sigma V^T
\]
We typically truncate this to the top $k$ components to return the rank-$k$ approximation.

\subsection{Theoretical Analysis and Proofs}
This section provides the mathematical justification for why the randomized approach yields an accurate approximation. We rely on the seminal analysis by Halko, Martinsson, and Tropp (2011).

\subsubsection{The Objective}
We want to bound the approximation error $\| A - Q Q^T A \|$ (where $\|\cdot\|$ denotes either the spectral norm $\|\cdot\|_2$ or Frobenius norm $\|\cdot\|_F$). Here, $P_Q = Q Q^T$ is the orthogonal projector onto the range of $Y = A\Omega$.

\subsubsection{Partitioning the Spectrum}
Let the exact SVD of $A$ be partitioned into the "target" $k$ components and the "tail":
\[
A = U \Sigma V^T = \begin{bmatrix} U_1 & U_2 \end{bmatrix} \begin{bmatrix} \Sigma_1 & 0 \\ 0 & \Sigma_2 \end{bmatrix} \begin{bmatrix} V_1^T \\ V_2^T \end{bmatrix}
\]
\begin{itemize}
    \item $U_1$ contains the top $k$ left singular vectors.
    \item $\Sigma_1$ contains $\sigma_1, \dots, \sigma_k$.
    \item $\Sigma_2$ contains the tail singular values $\sigma_{k+1}, \dots, \sigma_{\min(m,n)}$.
\end{itemize}
Ideally, the range of our test matrix $Q$ would align perfectly with $U_1$.

\subsubsection{Projection Analysis}
Let $\Omega$ be partitioned to match $V$:
\[
\Omega_1 = V_1^T \Omega \quad \text{and} \quad \Omega_2 = V_2^T \Omega
\]
The sketch matrix $Y$ can be written in the basis of $U$:
\[
Y = A \Omega = U \Sigma V^T \Omega = U_1 \Sigma_1 \Omega_1 + U_2 \Sigma_2 \Omega_2
\]
Intuitively, if $\Sigma_2$ is small (fast decay) and $\Omega_1$ is well-conditioned (which Gaussian matrices are, with high probability), then the range of $Y$ is dominated by $U_1$.

\subsubsection{The Deterministic Error Bound}
If $\Omega_1$ has full row rank (which is true with probability 1 since $\ell \ge k$), the error can be structurally bounded by:
\[
\| A - Q Q^T A \|^2 \le \| \Sigma_2 \|^2 + \| \Sigma_2 \Omega_2 \Omega_1^{\dagger} \|^2
\]
where $\Omega_1^{\dagger}$ is the Moore-Penrose pseudoinverse of $\Omega_1$.

The first term $\| \Sigma_2 \|$ is the theoretically optimal error (the baseline).
The second term represents the "penalty" for using randomization. It depends on the tail energy $\Sigma_2$ and the ratio of the norms of the random blocks.

\subsubsection{Probabilistic Average-Case Bound}
Taking the expectation over the Gaussian distribution of $\Omega$, we obtain specific bounds.

\begin{theorem}{Average Spectral Error}{avg_spec_err}
For a target rank $k$ and oversampling parameter $p \ge 2$:
\[
\mathbb{E} \| A - A_k^{\text{rsvd}} \|_2 \le \left( 1 + \frac{4\sqrt{k+p}}{p-1} \sqrt{\min(m,n)} \right) \sigma_{k+1}
\]
While this looks complex, the intuition is simpler. As we increase the oversampling $p$, the error factor approaches 1 rapidly.
\end{theorem}

\begin{theorem}{Average Frobenius Error}{avg_frob_err}
\[
\mathbb{E} \| A - A_k^{\text{rsvd}} \|_F \le \left( 1 + \frac{k}{p-1} \right)^{1/2} \left( \sum_{j=k+1}^{\min(m,n)} \sigma_j^2 \right)^{1/2}
\]
\end{theorem}

\subsubsection{The Role of Power Iterations}
In practice, if the singular values do not decay rapidly (the spectrum is flat), the mixing of $U_1$ and $U_2$ in the sketch $Y$ is strong. To fix this, we apply power iterations.

We replace $Y = A\Omega$ with:
\[
Y^{(q)} = (AA^T)^q A \Omega
\]
Analytically, this is equivalent to sketching the matrix:
\[
B = A (A^T A)^q
\]
The singular values of this matrix are $\sigma_j^{2q+1}$.

If the ratio between the signal and noise singular values was $\frac{\sigma_k}{\sigma_{k+1}}$, after $q$ iterations, the ratio becomes $(\frac{\sigma_k}{\sigma_{k+1}})^{2q+1}$.
Even a small gap is amplified exponentially, forcing the range of $Y$ to align almost perfectly with $U_1$.

The error bound with $q$ iterations becomes:
\[
\mathbb{E} \| A - A_{k,q}^{\text{rsvd}} \|_2 \le \left( 1 + \epsilon \right)^{1/(2q+1)} \sigma_{k+1}
\]
As $q$ increases, the error converges to the theoretically optimal $\sigma_{k+1}$.

\subsection{Complexity Analysis}
A rigorous analysis of computational complexity demonstrates why RSVD is superior to deterministic methods for low-rank approximations. We analyze the Floating Point Operations (FLOPs) required.

\textbf{Assumptions:}
\begin{itemize}
    \item Matrix $A$ has dimensions $m \times n$.
    \item We assume $m \ge n$ (if $n > m$, the analysis is symmetric).
    \item Target rank $k$ and oversampling $p$ define the sketch size $\ell = k + p$.
    \item We assume $\ell \ll n$, meaning we are looking for a significant dimension reduction.
\end{itemize}

\subsubsection{Step-by-Step Cost Breakdown}
\begin{itemize}
    \item \textbf{Generation of Test Matrix ($\Omega$):}
    Generating an $n \times \ell$ matrix of Gaussian random variables.
    Cost: $O(n\ell)$.
    Note: This is computationally negligible compared to matrix multiplication.

    \item \textbf{Range Sampling ($Y = A\Omega$):}
    This is a dense matrix multiplication of $(m \times n)$ by $(n \times \ell)$.
    Cost: $C_{\text{mult}} = 2mn\ell$ FLOPs.
    Significance: This is usually the dominant cost in the "one-pass" algorithm (where $q=0$).

    \item \textbf{Orthonormalization ($Y = QR$):}
    QR factorization of an $m \times \ell$ matrix using Householder reflectors or Modified Gram-Schmidt.
    Cost: $C_{\text{qr}} \approx 2m\ell^2$ FLOPs.
    Significance: Since $\ell \ll n$, this term $m\ell^2$ is much smaller than $mn\ell$.

    \item \textbf{Projection ($B = Q^T A$):}
    Matrix multiplication of $(\ell \times m)$ by $(m \times n)$.
    Cost: $C_{\text{proj}} = 2mn\ell$ FLOPs.

    \item \textbf{SVD of Small Matrix ($B$):}
    SVD of an $\ell \times n$ matrix.
    Cost: $C_{\text{svd}} \approx O(n\ell^2)$.

    \item \textbf{Lifting ($U = Q \tilde{U}$):}
    Matrix multiplication of $(m \times \ell)$ by $(\ell \times \ell)$.
    Cost: $O(m\ell^2)$.
\end{itemize}

\subsubsection{Total Complexity Comparison}
Summing the leading terms, the total cost for Basic RSVD ($q=0$) is:
\[
T_{\text{RSVD}} \approx 4mn\ell + O(m\ell^2 + n\ell^2)
\]
Since $\ell \ll n$, the $O(mn\ell)$ term dominates.

\textbf{Comparison with Classical Deterministic SVD:}
Golub-Reinsch SVD: Requires bidiagonalization followed by QR iteration.
Cost: $O(mn^2)$.

\textbf{Speedup Factor:}
\[
\frac{\text{Classical}}{\text{RSVD}} \approx \frac{mn^2}{mn\ell} = \frac{n}{\ell}
\]
If $n = 10,000$ and we want the top $k=100$ components (with $\ell \approx 110$), RSVD is roughly 100x faster.

\subsubsection{Complexity with Power Iterations}
If we perform $q$ power iterations to improve accuracy, we perform $q$ additional multiplications by $A$ and $A^T$.
Each iteration adds: $2 \times (mn\ell)$ FLOPs.

\textbf{Total Cost ($q > 0$):}
\[
T_{\text{RSVD}}^{(q)} \approx 2(2q + 2)mn\ell = O(qmn\ell)
\]
Even with $q=2$ or $q=3$, the algorithm remains linear in $m$ and $n$, preserving the massive advantage over the cubic/quadratic scaling of classical SVD.

\subsection{Approximation Quality and Error Intuition}
While Section 3.3 provided the raw probability bounds, this section focuses on the Eckart-Young-Mirsky theorem and the geometric intuition of why the error behaves as it does.

\subsubsection{The Baseline: Eckart-Young-Mirsky Theorem}
This theorem establishes the fundamental limit of low-rank matrix approximation. For any matrix $A$ and any matrix $A_k$ of rank at most $k$:
\[
\min_{\text{rank}(X) \le k} \| A - X \|_2 = \sigma_{k+1}
\]
No algorithm can beat this error in the spectral norm. The goal of RSVD is to produce an approximation $\hat{A}_k$ such that:
\[
\| A - \hat{A}_k \|_2 \approx \sigma_{k+1}
\]

\subsubsection{Interpretation of Oversampling ($p$)}
The oversampling parameter $p$ acts as a safety buffer.

\textbf{The Problem:} If we pick exactly $k$ random vectors, there is a non-zero probability that one of our random vectors falls orthogonal to one of the top $k$ principal components (singular vectors) of $A$. If this happens, we "miss" that component entirely.

\textbf{The Solution:} By selecting $\ell = k + p$ vectors, we span a slightly larger subspace. The probability that none of the $\ell$ random vectors capture the energy of the $i$-th singular vector decreases exponentially with $p$.

\textbf{Result:} A small $p$ (e.g., $p=5$ or $p=10$) is sufficient to make the failure probability negligible ($< 10^{-7}$).

\subsubsection{Intuition for Power Iterations ($q$)}
Power iterations are necessary when the singular value spectrum decays slowly (the "flat spectrum" problem).
Consider the ratio of the "signal" (the $k$-th singular value) to the "noise" (the $(k+1)$-th singular value).
Let $\gamma = \frac{\sigma_{k+1}}{\sigma_k}$.
\begin{itemize}
    \item If $\gamma \approx 0$ (fast decay), the signal is distinct.
    \item If $\gamma \approx 1$ (slow decay), the signal is hard to distinguish from the tail.
\end{itemize}
When we compute $Y = (AA^T)^q A \Omega$, we are effectively sampling from a matrix with singular values $\sigma_i^{2q+1}$.
The new ratio becomes:
\[
\gamma_{\text{new}} = \left( \frac{\sigma_{k+1}}{\sigma_k} \right)^{2q+1}
\]
\textbf{Example:}
If $\sigma_k = 1.0$ and $\sigma_{k+1} = 0.9$ (a difficult case):
\begin{itemize}
    \item $q=0$: Ratio is $0.9$. Separation is poor.
    \item $q=2$: Exponent is 5. Ratio is $(0.9)^5 \approx 0.59$. Separation improves.
    \item $q=5$: Exponent is 11. Ratio is $(0.9)^{11} \approx 0.31$. Separation is excellent.
\end{itemize}
By artificially sharpening the decay of the singular values, we force the random vectors to align with the top $k$ singular vectors much faster.

\subsection{Implementation Details (Python)}
Implementing RSVD requires attention to numerical stability, specifically regarding the precision of floating-point operations and the order of matrix multiplications.

\subsubsection{Key Design Choices}
\begin{itemize}
    \item \textbf{BLAS Level 3 Operations:} In Python, using \texttt{numpy.dot} or the \texttt{@} operator calls underlying BLAS (Basic Linear Algebra Subprograms) routines. These are highly optimized for modern CPU caches.
    \item \textbf{QR Factorization Mode:} We use \texttt{mode='reduced'} (or economic). We do not want the full $m \times m$ Q matrix, only the $m \times \ell$ active columns.
    \item \textbf{Numerical Stability:} For the Power Iteration, repeating $Y = A(A^T Y)$ is susceptible to round-off error if $q$ is large. In very high-precision requirements, one might orthonormalize between iterations, but for standard ML applications ($q<5$), direct multiplication is sufficient and faster.
\end{itemize}

\subsubsection{Python Implementation Code}
\begin{verbatim}
import numpy as np

def rsvd(A, k, p=10, q=2):
    """
    Computes the Randomized SVD of matrix A.
    
    Args:
        A (np.ndarray): Input matrix of shape (m, n).
        k (int): Target rank.
        p (int): Oversampling parameter.
        q (int): Number of power iterations.
        
    Returns:
        U_k (np.ndarray): Top k left singular vectors (m, k).
        S_k (np.ndarray): Top k singular values (k,).
        Vt_k (np.ndarray): Top k right singular vectors (k, n).
    """
    m, n = A.shape
    ell = k + p
    
    # 1. Generate Random Test Matrix
    # We use standard normal distribution.
    Omega = np.random.randn(n, ell)
    
    # 2. Sample the Range (with Power Iteration)
    # Initial projection
    Y = A @ Omega
    
    # Power Iterations to separate signal from noise
    for _ in range(q):
        Y = A @ (A.T @ Y)
    
    # 3. Orthonormalization (QR Decomposition)
    # 'reduced' ensures Q is (m, ell), not (m, m)
    Q, _ = np.linalg.qr(Y, mode='reduced')
    
    # 4. Project A into low-dimensional subspace
    # B is small: (ell, n)
    B = Q.T @ A
    
    # 5. Deterministic SVD on small matrix B
    # full_matrices=False ensures we get compact SVD
    U_tilde, S, Vt = np.linalg.svd(B, full_matrices=False)
    
    # 6. Lift singular vectors back to original space
    U = Q @ U_tilde
    
    # 7. Truncate to rank k
    U_k = U[:, :k]
    S_k = S[:k]
    Vt_k = Vt[:k, :]
    
    return U_k, S_k, Vt_k

# Example Usage Verification
if __name__ == "__main__":
    # Create a synthetic low-rank matrix
    m, n = 1000, 500
    r = 20 # True rank
    
    # Generate random factors
    U_true = np.random.randn(m, r)
    V_true = np.random.randn(n, r)
    # Force orthogonality for clearer singular values
    U_true, _ = np.linalg.qr(U_true)
    V_true, _ = np.linalg.qr(V_true)
    
    # Singular values decay
    S_true = np.linspace(100, 1, r)
    
    # Construct A
    A = (U_true * S_true) @ V_true.T
    
    # Add some noise
    A += 0.01 * np.random.randn(m, n)
    
    # Run RSVD
    k_target = 20
    U_approx, S_approx, Vt_approx = rsvd(A, k=k_target, p=10, q=2)
    
    # Check reconstruction error
    A_approx = (U_approx * S_approx) @ Vt_approx
    error = np.linalg.norm(A - A_approx) / np.linalg.norm(A)
    print(f"Relative Reconstruction Error: {error:.5f}")
\end{verbatim}

\subsubsection{Memory Management Note}
For extremely large matrices that do not fit in RAM, standard NumPy cannot be used. In such cases, this algorithm can be adapted for Out-of-Core processing or distributed systems (e.g., PySpark or Dask). The logic remains identical, but the matrix multiplications $A\Omega$ and $Q^T A$ are performed by streaming blocks of $A$ from disk.

\subsection{Experimental Workflow for RSVD}
To rigorously validate the theoretical claims of RSVD, we must design an experimental framework that isolates the effects of rank ($k$), oversampling ($p$), power iterations ($q$), and matrix structure (spectral decay).

\subsubsection{Dataset Generation (Matrix Families)}
The performance of RSVD depends heavily on the singular value spectrum of the input matrix. We categorize test matrices into three distinct families:

\textbf{Fast Decay (Ideal Case):}
\begin{itemize}
    \item \textbf{Description:} Matrices where singular values decay exponentially ($\sigma_j \approx e^{-\alpha j}$).
    \item \textbf{Source:} Real-world datasets like "Eigenfaces" (face recognition), user-item recommendation matrices, and certain physical simulations (e.g., heat diffusion).
    \item \textbf{Expectation:} RSVD should achieve near-optimal error with $q=0$ or $q=1$.
\end{itemize}

\textbf{Slow Decay (The "Heavy Tail"):}
\begin{itemize}
    \item \textbf{Description:} Matrices where singular values decay polynomially ($\sigma_j \approx j^{-1}$).
    \item \textbf{Source:} Large-scale graph Laplacians (social networks), term-document matrices in NLP (Zipf's law).
    \item \textbf{Expectation:} Basic RSVD ($q=0$) will likely perform poorly compared to deterministic SVD. This is the critical test case for Power Iterations ($q \ge 2$).
\end{itemize}

\textbf{Structured Random Matrices (Worst Case):}
\begin{itemize}
    \item \textbf{Description:} Dense Gaussian matrices scaled by magnitude.
    \item \textbf{Source:} Synthetic noise benchmarks.
    \item \textbf{Expectation:} These have a "flat" spectrum where $\sigma_k \approx \sigma_{k+1}$. RSVD helps with speed, but accuracy gains are harder to visualize because the "true" low-rank approximation is mathematically ill-defined (no clear separation between signal and noise).
\end{itemize}

\subsubsection{Metrics for Evaluation}
We quantify performance using two primary dimensions: Accuracy and Efficiency.

\textbf{1. Relative Reconstruction Error:}
We measure how close the RSVD approximation is to the original matrix relative to its energy.
\[
E_{\text{rel}} = \frac{\| A - U_k \Sigma_k V_k^T \|_F}{\| A \|_F}
\]
Why Frobenius norm? It is computationally cheaper to estimate than the spectral norm for large matrices.

\textbf{2. Approximation Ratio (Competitive Analysis):}
We compare RSVD against the "Gold Standard" (deterministic truncated SVD computed via ARPACK/scipy.sparse.linalg.svds).
\[
\rho = \frac{\| A - A_k^{\text{rsvd}} \|_F}{\| A - A_k^{\text{optimal}} \|_F}
\]
\begin{itemize}
    \item $\rho = 1.0$: RSVD is perfect.
    \item $\rho \approx 1.05$: RSVD is within 5\% of the optimal error.
\end{itemize}

\textbf{3. Speedup Factor:}
\[
S = \frac{T_{\text{deterministic}}}{T_{\text{rsvd}}}
\]
We measure wall-clock time, ensuring we exclude the time taken to generate the synthetic data.

\subsubsection{Experimental Procedure}
For each matrix family and size $N \in \{1000, 5000, 10000\}$:
\begin{itemize}
    \item \textbf{Baseline:} Compute Exact SVD once to get ground truth singular values and vectors.
    \item \textbf{Grid Search:} Run RSVD varying:
    \begin{itemize}
        \item Target rank $k \in \{10, 50, 100\}$.
        \item Oversampling $p \in \{0, 5, 10, 20\}$.
        \item Power Iterations $q \in \{0, 1, 2, 4\}$.
    \end{itemize}
    \item \textbf{Plotting:}
    \begin{itemize}
        \item \textbf{Plot 1 (Convergence):} Error vs. Oversampling ($p$). Hypothesis: Error drops sharply until $p=10$, then plateaus.
        \item \textbf{Plot 2 (Robustness):} Error vs. Power Iterations ($q$) for "Slow Decay" matrices. Hypothesis: Significant error reduction from $q=0$ to $q=2$, diminishing returns afterwards.
        \item \textbf{Plot 3 (Scalability):} Runtime vs. Matrix Size ($n$). Hypothesis: RSVD scales linearly; Deterministic scales super-linearly.
    \end{itemize}
\end{itemize}

\subsection{Practical Applications of RSVD}
RSVD is not just a theoretical improvement; it is the engine behind many modern large-scale machine learning systems.

\subsubsection{Dimensionality Reduction (PCA)}
Principal Component Analysis (PCA) is mathematically equivalent to SVD on the centered covariance matrix.
\begin{itemize}
    \item \textbf{Challenge:} In genomics (e.g., single-cell RNA sequencing), we often have matrices with $10^5$ genes (features) and $10^5$ cells (samples). Computing full covariance is $O(n^2)$, which is impossible.
    \item \textbf{RSVD Solution:} We can apply RSVD directly to the data matrix $X$ to get the top principal components without ever forming the covariance matrix $X^T X$. This allows PCA on massive biological datasets.
\end{itemize}

\subsubsection{Image Compression and Denoising}
\begin{itemize}
    \item \textbf{Compression:} An image is a matrix of pixel intensities. By keeping only the top $k=50$ singular values, we can store a high-resolution image with 10-20\% of the original file size.
    \item \textbf{Denoising:} Noise in images typically manifests as high-frequency, low-energy variations. These naturally fall into the "tail" of the singular value spectrum. RSVD acts as a filter: by reconstructing the image using only the top $k$ components, we inherently discard the noise (which lives in the range of $\sigma_{k+1} \dots \sigma_n$).
\end{itemize}

\subsubsection{Latent Semantic Analysis (LSA) in NLP}
\begin{itemize}
    \item \textbf{Context:} Document-Term matrices (Bag of Words) are massive and sparse.
    \item \textbf{Application:} RSVD extracts "topics" (singular vectors) representing clusters of related words. Because RSVD only requires matrix-vector products, it can exploit the sparsity of the input matrix ($A\Omega$ is fast if $A$ is sparse), making it far more efficient than dense SVD solvers.
\end{itemize}

\subsubsection{Pre-processing for Matrix Multiplication}
This is the specific use case for your broader project.
\begin{itemize}
    \item \textbf{Problem:} We want to multiply two massive matrices $A \times B$.
    \item \textbf{Strategy:}
    \begin{enumerate}
        \item Approximate $A \approx U_A \Sigma_A V_A^T$ using RSVD (rank $k$).
        \item Approximate $B \approx U_B \Sigma_B V_B^T$ using RSVD (rank $k$).
        \item Compute $A \times B \approx (U_A \Sigma_A V_A^T) (U_B \Sigma_B V_B^T)$.
    \end{enumerate}
    \item \textbf{Benefit:} Instead of $O(n^3)$ operations, we perform multiplications on the smaller factor matrices ($O(nk^2)$), achieving massive speedups when $k \ll n$.
\end{itemize}

\subsection{Conclusion}
Randomized SVD represents a paradigm shift in numerical linear algebra. It moves away from the 20th-century focus on deterministic, worst-case guarantees (like full QR iteration) toward a probabilistic, average-case perspective suitable for the Big Data era.

\subsubsection{Summary of Key Findings}
\begin{itemize}
    \item \textbf{Efficiency:} RSVD reduces the complexity of low-rank approximation from cubic $O(mn^2)$ to linear $O(mnk)$ (in terms of input size). This makes it feasible to process matrices that were previously intractable.
    \item \textbf{Accuracy:} Theoretical bounds and empirical evidence confirm that with modest oversampling ($p=10$) and minimal power iterations ($q=2$), RSVD produces approximations that are indistinguishable from the optimal truncated SVD for all practical purposes.
    \item \textbf{Simplicity:} The algorithm is structurally simple—relying primarily on matrix multiplication and standard QR/SVD on small matrices. This makes it highly parallelizable and suitable for GPU acceleration.
\end{itemize}

\end{document}
